% neomake: skip
\documentclass{article}

\usepackage[utf8]{inputenc}
\usepackage[english,russian]{babel}
\usepackage{tikz}
\usepackage{amsmath}
\usepackage{amssymb}
\usepackage{multicol}
\usepackage{lipsum}
\usepackage{hyperref}
\usepackage{graphicx}
\usepackage{xcolor}

\usepackage{bbding}
\newcommand*\tick{\item[\Checkmark]}
\newcommand*\fail{\item[\XSolidBrush]}

\usepackage{geometry}
\geometry{top=15mm}
\geometry{bottom=25mm}
\geometry{left=10mm}
\geometry{right=35mm}

\renewcommand{\thesection}{\Alph{section}:}
\renewcommand{\thesubsection}{$\Alph{section}_\arabic{subsection}$:}
\renewcommand{\thesubsubsection}{$\Alph{section}_{\arabic{subsection}.\Roman{subsubsection}}$:}
\renewcommand{\thepage}{Страница примерно \Roman{page}}

\usepackage{tikz}

\begin{document}
    \tableofcontents
    \section{Это секция}
    \begin{minipage}[t]{0.3\textwidth}
        $$\mathfrak{Mot\ddot{o}rhead}$$
    \end{minipage}
    \begin{minipage}[t]{0.5\textwidth}
        \begin{equation}
            \label{eq1}
            f(x) = \left\{
            \begin{array}{rc}
                x + 2, & x \geqslant 1\\
                -x,    & x < 1
            \end{array}
            \right.
        \end{equation}
    \end{minipage}

    \subsection{Популярные видео}
        \begin{multicols}{3}
            \framebox{
                \vbox to 13em {
                    \begin{itemize}
                        \tick Эта чешет колоду
                        \fail Этот стоит
                        \tick
                            \begin{enumerate}
                                \item Колода была заряжена в киоске
                                \begin{enumerate}
                                    \item Карты разложены по-другому
                                    \item [3.14] $\pi$
                            \end{enumerate}
                        \end{enumerate}
                    \end{itemize}
                }
            }
            \begin{flushright}
                \colorbox{yellow!20}{
                    \textbf{
                        \textcolor{red}{Я}
                        \textcolor{green}{еду}
                        \textcolor{magenta}{в}
                        \textcolor{blue}{стройку}
                    }
                }
            \end{flushright}
            \vbox to 13em {
                \includegraphics[height=10em]{pic.jpg}
            }
        \end{multicols}
        \subsubsection{Ссылки}
            Это ссылка на формулу \ref{eq1}
        \subsubsection{tikz}

            \begin{multicols}{2}
                \begin{tikzpicture}
                    \draw[rotate=0] (0,0) ellipse (100pt and 50pt);
                    \draw (-1,0) to[out=30,in=150] (1,0);
                    \draw (-1.2,.1) to[out=-30,in=-150] (1.2,.1);
                \end{tikzpicture}

                Эллипс у этого тора имеет центр в нуле, а его радиусы -- $100pt$ и $50pt$.
                Кривые начинаются из $-1$ и $-1.2$ и заканчиваются в $1$ и в $1.2$ соответственно.
            \end{multicols}

            \begin{multicols}{2}
            \begin{tikzpicture}
                  \draw[dotted] (-0.5, -3) grid (3.5, 2.5);

                  \draw[black, ultra thick, ->] (-0.5, 0) -- (3.5, 0) node[below]{$x$};
                  \draw[black, ultra thick, ->] (0, -3) -- (0, 2.5) node[left]{$y$};
                  \foreach \x in {1,...,3} \node[below] at (\x, 0){$\x$};
                  \foreach \y in {-2,...,2} \node[left] at (0, \y){$\y$};

                  \draw[black, line width = 1mm, green!80!blue]   plot[smooth, domain=0:2.5] (\x, 0.5 * \x * \x);
                  \fill[green!80!blue] (2, 2) circle (0.2) node[below, outer sep = 5pt, black]{$(x, y)$};
            \end{tikzpicture}

            \Huge
            $y = 0.5 x^2$
            \end{multicols}

\end{document}
