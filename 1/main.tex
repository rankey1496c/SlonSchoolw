% neomake: skip
\documentclass{article}

\usepackage[utf8]{inputenc}
\usepackage[english,russian]{babel}
\usepackage{tikz}
\usepackage{amsmath}
\usepackage{amssymb}
\usepackage{multicol}
\usepackage{lipsum}


\newcommand{\starcup}{$\sqcup$\kern-0.58em{$\star$}}

\newcommand{\foo}[1]{
    \begin{tikzpicture}[#1]
        \draw (0,0) -- (1ex,1ex);
    \end{tikzpicture}
}

\begin{document}
    \section{Это секция}
    {\LARGE Здравствуйте дорогой Мартин Алексеич!}\kern-30.5em
    {\footnotesize Пишу вам сразу по приезду, прямо вот только что вошёл и сел писать.}

    \begin{flushright}
        $
        \boldsymbol{
            \left\langle
            \frac{1\backslash2}{34}\Big|_0^2
            \right)
        }
        $
    \end{flushright}

    \subsection{$\alpha$}
    \textbf{\textit{Лабиринт}} (др.-греч. $\lambda\alpha\beta\upsilon\rho\iota\nu\theta o \varsigma$) ---
    какая-либо структура (обычно в двухмерном или трёхмерном пространстве), состоящая из запутанных путей
    к выходу (и/или путей, ведущих в тупик). 
    \subsubsection{$\beta$}

    \begin{multicols}{3}

    $$\log\left(
    \prod_{i=1}^{n} f_i(x)
    \right) =
    \sum\limits_{i=1}^{n} \log f_i(x)
    $$

    $$\pmb{\vec{a} = \frac{d\vec{v}}{dt}}$$

    $$f'(x_0) = \lim_{x \to x_0} \frac{f(x) - f(x_0)}{x - x_0}$$
    \end{multicols}

    \subsubsection{$\Omega$}
    \begin{multicols}{2}
        \lipsum[1-2]
    \end{multicols}


\end{document}
